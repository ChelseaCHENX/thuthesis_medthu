\% !TeX root = ../thuthesis-example.tex
\thusetup{
  cite-style = author-year,
}
\noindent
\chapter{Introduction}

\section{Motivation}

Cancer is the leading cause of lethality worldwide, as a byproduct of modern-world longevity – it arises from naturally accumulating mutations, which eventually lead to uncontrollable growth of malignant cells. Cancer is generally classified by tissues of origin, but past decades of molecular biology development have enabled recognition of finer subtypes - each with distinct phenotypes, prognosis, and thus specific treatment. Tumor heterogeneity (TH) emphasizes such variations - as differences between (inter) or within (intra) tumors. It develops via two factors – genomic / epigenetic aberrations (intrinsic), and microenvironment signals (extrinsic) - mutants appear intrinsically and evolve under selective pressure, while successfully expanded clones recruit vessels or suppress immune activities, and thus in turn modify the environment. Given such dynamic heterogeneity, this resultant repertoire may well preserve clones with high invasiveness  to cause metastasis, or high adaptation to cause resistance -altogether posing great challenges to cancer treatment.

Genomic or epigenomics factors are the most recognized contributors to TH. Genetic aberrations result mostly from genomic instability – aberrant ploidy, structural variants, indels or point mutations, arising from mitosis errors or DNA repair deficiency. Among numerous mutations, the specific group of ‘drivers’ are of particular interest, as they greatly promote cancer growth, proliferation or invasiveness – e.g., by hyper-activation of oncogenes (e.g., PI3Ks, MYC, CDKs) and loss of function of tumor suppressor genes (TSG) (e.g., TP53, RB1). Epigenetic states are influenced by both genetic mutations (e.g., chromatin modifiers, transcriptional factors or co-factors) or environment – and are shown as hyper/hypo histone methylation/acetylation, altered 3D chromosome structure, and different chromatin accessibility. These may cause aberrant gene expression - e.g. TSG (CDKN2A, APC, MLH1) silencing by increased promoter CpG and H3K9 methylation, and oncogene (c-Myc, MAGE, CDH3) activation by decreased methylation at promoter CpG islands \citep{cheung2009dna}. Besides these two factors which are relatively stable, cells also render great plasticity, in case of stimulations in particular. These responses are also largely heterogeneous – as reflected by transcriptomic or proteomics profiles, and can equally influence cancer adaptation and evolution as genetic/epigenetic factors.

Multiple techniques are available in measuring tumor heterogeneity – the most well-known example being large cohort studies like TCGA, which used microarrays or bulk RNA sequencing to identify molecular subtypes; as well as the large scale DNA sequencing by ICGC (International Cancer Genome Consortium) or PCAWG (Pan-Cancer Analysis of Whole Genome). Moreover, epigenetics can also be profiled - ChIP-seq identifies sequence bound to histone markers (H3K27ac, H3K27me3, H3K4me1, H3K4me3, H3K9me3) or other transcriptional factors of interest; methylation states can be detected with bisulfate sequencing (WGBS (whole-genome bisulfate sequencing), RRBS (reduced-representation bisulfite sequencing)) or microarray (Illumina Infinium BeadChip); while DNAase-I-seq or ATAC-seq can be used to assess chromatin accessibility. Expression and of total and phosphorylated proteins can be quantified as well – e.g., by RPPA (reverse phase protein array) by using hundreds of antibodies. However, these measurements only resolve phenotypes at tumor level as a mixture of all tumor and/or TME components, in which case intra-tumor heterogeneity (ITH) is not directly explored, and might influence bulk sequencing due to purity issues. To solve this problem, multiple computational methods have been developed, e.g., to dissect clonal states from the bulk DNA-seq data \citep{roth2014pyclone}, SubcloneSeeker \citep{qiao2014subcloneseeker}, or to deconvolute bulk RNA-seq (CIBERSORT \citep{chen2018profiling}, ESTIMATE \citep{yoshihara2013inferring}). But they are nevertheless limited by performance uncertainty which may be data specific. On the other hand, single cell sequencing has developed rapidly during the past decade, for either single omic (scDNA-seq, scATAC-seq, scRNA-seq, CyTOF \citep{bendall2011single}), or multi-omics within the same cell (DR-seq \citep{dey2015integrated}, G\&T-seq \citep{macaulay2015g}, scTrio-seq \citep{hou2016single}, scMT-seq \citep{hu2016simultaneous}) – these techniques thus greatly promote ITH research. 

The ultimate goal of heterogeneity characterization is to suggest personalized therapies targeting pathogenic subclones with driver mutations or transcriptomic signatures, which could then be tested in in vitro (organoids) or in vivo (xenografts) preclinical models. These features can also be traced longitudinally by liquid biopsy, to monitor recurrence, assess responsiveness and guide further decisions.

In summary, tumor heterogeneity extensively exists and greatly contributes to cancer resistance and mortality. Bulk or single cell sequencing can be used for studying inter and intra TH, to identify pathognomonic mechanisms and suggest personalized therapies. Towards this direction, great efforts and achievements have been made, especially in breast cancer - these will be elaborated in the later sections.

\section{Background and Significance}

Breast cancer (BC) is the most common cancer in women worldwide, and the second leading cause of cancer mortality in women in the United States (US). In 2019, there were over 270,000 new diagnosis and more than 42,000 deaths in US. BC is therefore an essential public health issue calling for better prevention, diagnosis and treatment strategies.

During the past decades, research and treatment of BC have both greatly developed, with survival tremendously improved. Current management of primary BC starts with TNM staging and subtyping at diagnosis. Subtyping usually include histopathology and marker expression (ER (estrogen receptor), PR (progesterone receptor), HER2 (human epidermal growth factor receptor 2)) by IHC (immunohistochemistry). Molecular diagnostic by microarray is also applicable to evaluate risk of recurrence (e.g., MammaPrint \citep{cardoso201670}. Regarding treatment, surgery removal is still a primary option, followed by adjuvant therapeutics of endocrine, chemo, targeting, or immune therapies, based on diagnostic features accordingly.

Most (70\%) early-staged primary BC never relapse after surgery and adjuvant therapy. However, for those 30\% who do relapse, distant dissemination composed of multiple subclones usually occurs, followed by the rapid development of resistance and eventually mortality, thus making therapeutics particularly difficult. Precision medicine could suggest potential solutions, including inhibiting druggable pathogenic mutations or signaling signatures, immuno-targeting tumor-specific antigens, and the sequential administration of combinatorial therapies. The following sections will demonstrate breast cancer subtyping, metastasis, resistance, and treatment.


\subsection{Breast Cancer Subtypes}

Most breast cancers are often identified by mammography screening (2/3) or palpation (1/3), and further confirmed from biopsy \citep{caughran2018effect}. Various factors are considered in diagnosis: histological grade (e.g., Nottingham Prognostic Index), TNM staging, proliferative activity, subtypes, and risk factors. Breast cancer subtyping is the basis of many subtype-specific treatment, mainly consisting of three classification systems - histology, molecular pathology and expression profiles. 

Histological subtypes are determined morphologically by tumor architecture. Similar to other carcinoma, BC consists of carcinoma in situ and invasive carcinoma, based on existence of basement membrane invasion. Both types mainly include ductal and lobular subtypes - while other subtypes (tubulo-lobular, medullary, mucinous, neuroendocrine and invasive papillary, etc.) also exist in invasive carcinoma with lower frequency and less characterization. IDCs (invasive ductal carcinoma) account for 70-80\% of all invasive BC, featuring growth of cell clusters. ILCs (invasive ductal carcinoma) accounts for another 5-15\%, most showing E-cadherin loss of expression, and exhibiting single-cell alignment from low intercellular adhesion \citep{american2019breast}. 

Molecular pathology is the expression of a few essential genes which determine disease behavior - e.g., in BC, being hormone receptors (ER, PR, HER2); and Ki67 as proliferation indicator, the expression of which are easily assessed by IHC or IF (immunofluorescence). Positivity is defined differently for each marker - e.g., >=1\% of stained tumor cells for ER or PR while >10\% positive cells for HER2 \citep{fragomeni2018molecular}. ER is a transcription factor (TF) canonically activated by estrogen and is one most essential indicator of endocrine therapy benefits. PR is also a steroid hormone receptor and TF, which can be induced by estrogen and in turn influence ER. Positivity in both ER and PR indicates good prognosis and endocrine responsiveness pathways \citep{lange2008challenges}. HER2 is a receptor tyrosine kinase (RTK) encoded by ERBB2 gene, which when activated, promotes survival, proliferation and invasiveness through activating the PI3K-Akt and Sos-Ras-MAPK pathways \citep{hudis2007trastuzumab}. ERBB2 amplification is detected in 15-30\% of breast cancers, and typically indicates high invasiveness. Anti-HER2 treatment is often suggested - trastuzumab, pertuzumab as monoclonal antibodies (mAb), and small molecules (gefitinib, lapatinib) targeting tyrosine kinase. Ki67 is expressed in active cell cycle, which generally reflects active proliferation and predicts poor outcomes. Chemotherapy is often suggested for tumors with high Ki67 level.

In contrast to a few markers in molecular pathology, expression profiling screens expressions of more genes, aiming for more precise phenotyping and prognostic prediction. The most canonically classification includes four subtypes: luminal A (LumA), luminal B (LumB), HER2 enriched (HER2) and basal like (Basal). This scheme traced back to 2000 - as clusters among 84 breast samples from expression of 496 ‘intrinsic genes’ proposed by Perou et al. \citep{perou2000molecular}. These genes include members regulating proliferation, ER and HER2 expression and those at chromosome 17. This was later curated as a 50 gene panel (Prediction Analysis of Microarray 50, PAM50), and commercialized as Prosigna signature assay by NanoString, to predict recurrence risk and endocrine/chemotherapy benefits \citep{wallden2015development}.  Both LumA and LumB are ER+ tumors, which derive their names by expression of normal luminal epithelial markers (KRT8/18). LumA has the most favorable prognosis, often of low grade (1 or 2), ER+/PR-high/HER2- and Ki-67-low phenotypes, and comprise of 30-40\% of the cases - and is often treated with 5-10 years of endocrine therapy alone \citep{fragomeni2018molecular}. LumB is similar to LumA as being ER+/HER2- while with lower PR expression, and differs from LumA in having high Ki67 index (>14\%) and recurrence risk. Combined chemo and endocrine therapies are suggested for LumB. HER2 and Basal subtypes are often ER negative - HER2 tumors show HER2 overexpression, mostly resulted from ERBB2 amplification, and are treated with anti-HER2 mAbs; Basal tumors are triple receptor negative (ER-/PR-/HER2-), often with high Ki-67 levels, express myoepithelial markers (KRT5/17, P-cadherin, EGFR), and showing high grade with poor differentiation \citep{weigelt2010histological}. It generally has the worst prognosis of all, while being high chemotherapeutically responsive. 

Apart from the PAM50 subtypes, alternative gene panels are also useful in predicting recurrence risk and chemotherapy benefits, the most recognized ones being MammaPrint and Oncotype DX. MammaPrint is a 70 gene microarray-test mainly implemented in Europe \citep{cardoso201670}. It is based on retrospective studies of patients with small (<5cm), node negative, low stage (I, II) tumors, which is further validated in two prospective trials (RASTER, MINDACT), and useful in predicting low or high risk of 5-year distal metastasis of early stage tumors. Oncotype DX is an qPCR assay of 21 genes popular in US \citep{paik2004multigene} which predicts a 10-year distant recurrence risk for patients with node-negative ER+ tumors. These molecular panels thus provide additional risk assessment and could assist decision making in further treatment.

Subtyping from the three systems are partially associated - molecular pathology largely corresponds to expression profiling, and some histological subtypes also preferentially overlap with particular molecular subtypes (e.g., most invasive lobular, tubular and mucinous carcinomas are luminal, and medullary or metaplastic carcinoma are basal.) Of note, the current PAM50 panel is derived mostly from IDCs, and thus may not fully apply to other rarer histology. Refined signatures for non-IDC subtypes were thus necessary - e.g., the three ILC mRNA subtypes (reactive-like, immune-related, proliferative) by TCGA \citep{ciriello2015comprehensive}, and the 194 prognostic panel of LobSig for ILC \citep{mccart2019lobsig}. It is expected that novel signatures will continue to develop with more sequencing clinical data collected, and hopefully be more personalized, accurate, and indicative. 


\subsection{Breast Cancer Treatment}

Treatment for BC majorly includes local (surgery, radiation) and systemic therapies (drugs). Primary BC mostly goes through surgical resection - neoadjuvant treatment is sometimes adopted to shrink tumor size, and adjuvant therapy often follows after surgery for 5-10 years. Radiation may also be used to suppress residual diseases. Metastatic BC is usually incurable - the therapeutics against which aim to prolong lifespan while palliating symptoms, and may also selectively include surgery/radiation/drugs. Drug treatment mainly include chemo and endocrine therapies as illustrated in Figure 1.1, while alternative choices like targeting and immune therapies may also apply in certain cases. The following sections will mainly introduce systemic therapies regarding their mechanisms and applications.

\subsubsection{Endocrine therapy}

70\% of invasive BC is ER positive. In these case, endocrine therapy is suitable as it blocks estrogen signaling-mediated tumor growth \citep{lumachi2013treatment}. Estrogen is mainly generated from ovary in premenopausal women, and other tissues (liver, muscle, adipose) in postmenopausal women. It is developed from androgens by aromatase (CYP19) and eventually converts into four metabolites - estrone (E1), estradiol (E2), estriol (E3) and estrone sulfate (E1S) - among which E2 stands for the active form with high ER-binding affinity \citep{lonning2011exploring}. ER (ESR1) normally resides as monomers within cytoplasm - upon binding with E2, it goes through conformational changes and becomes homodimerized, moves into nucleus and binds to specific DNA sequences (estrogen-response elements (EREs)). It then recruits other TFs or co-regulators into a complex, which altogether induces transcription of multiple downstream genes, including pathways promoting survival, growth, proliferation and invasion \citep{song2006membrane}. 

Three types of drugs are mostly used in endocrine therapies - selective estrogen receptor modulators (SERMs), selective estrogen receptor degraders (SERDs) and aromatase inhibitors (AIs). In breast tissue, SERMs (e.g., tamoxifen) act as competitive E2 antagonist - upon binding, it causes aberrant positioning of helix 12, which occludes the coactivator recognition groove and thus prevents further coactivator binding \citep{shiau1998structural}. However, as the ‘selective’ nomenclature suggests, such modulating effects vary across tissues - e.g., tamoxifen (Tam) acts as partial agonists in bone, cardiovascular system and uterine endometrium \citep{hodges2003tamoxifen}. This could be explained by differential expression of co-receptors across tissues - e.g. co-suppressors in breast, but co-activators (SRC-1, AIB1, CBP) in endometrium  \citep{shang2002molecular}. Tamoxifen (Tam) is the most commonly used SERM, which functions through its active metabolite, 4-hydroxytamoxifen. It gained attention in 1970s by showing comparable efficacy but greatly decreased side effects than the high-dose estrogen regimen popular at that time, and continued such success in clinical trials for the following decades, showing efficacy in both early and advanced diseases \citep{patel2018selective}. Other FDA-approved SERMs include toremifene - in both BC prevention and treatment, and raloxifnene in BC prevention. SERDs similarly target estrogen signaling but with different mechanism as to destruct ER upon binding. Fulvestrant is among the most well-known SERD (ICI 182,780), which was developed from ICI 164,384 and shows potency in inhibiting tamoxifen-resistant BC progression. It, however, has low bioavailability and requires intramuscular injection, which prompts investigation of oral SERDs as alternatives - e.g., GW5638, GDC-0810, AZD9496, and RAD1901 \citep{mcdonnell2015oral}. Additional to SERMs and SERDs, AIs can also block estrogen signaling by inhibiting estrogen synthesis. It shows great effectiveness in post-menopausal women, but with higher adverse effects than Tam. These include compounds like anastrozole, letrozole, and exemestane \citep{smith2003aromatase}. In general, endocrine therapy is recommended as adjuvant therapy for all ER/PR+ patients for five years or beyond - and its combination with other drugs are backed on other clinicopathological / molecular evidences.

\subsubsection{Anti-HER2 therapy}

Anti-HER2 therapies greatly improved outcomes of ERBB2 positive tumors - in the combined analysis of two trials of chemotherapy with or without trastuzumab, 10-year relapse free survival (RFS) reaches 81.39\%than 65.17\%, and 77.78\% than 59.27\%, in the combined group versus chemotherapy alone, for HR positive and negative patients respectively \citep{chumsri2019incidence}. HER2 is a receptor tyrosine kinase of the HER family encoded by ERBB2. A complete transmembrane HER2 protein can dimerize with another ligand-bound HER (1,3,4) partner, cross phosphorylate tyrosine residues in their intracellular domain, and eventually promote tumor proliferation, facilitate invasiveness, and induce angiogenesis by VEGF expression increasement. Alternatively, HER2 could also present as a truncated form - an intracellular 95 residues from C-terminal (p95), resulted from alternative translation initiation or proteolytic cleavage. This variant readily causes constant activation of the intracellular domain and similar downstream pathways \citep{hudis2007trastuzumab}. 

Trastuzumab came up on stage in four trials in 1990s, which greatly improved survival of ERBB2+ patients when combined with chemotherapy. It has currently become the standard of care to use paclitaxel / trastuzumab for small, node-negative ERBB2+ tumors \citep{tolaney2015adjuvant}. Trastuzumab interferes with HER2 mediated signals mainly in three ways - 1) creating physical occupancy to hider cleavage or dimerization mediated activation, 2) induce receptor endocytosis upon binding, and 3) attract immune elimination of cancer cells. The third mechanism benefits from its delicate structure - two antigen binding sites which specifically recognizes HER2, and a humanized Fc region recognizable by FcR immune effector cells - and thus induces ADCC (antibody-dependent, cell-mediated cytotoxicity) \citep{hudis2007trastuzumab}. Pertuzumab is another mAb which targets dimerization domain of HER2. It was showed to increase 3-year invasive disease free survival (DFS) when combined with trastuzumab and chemotherapy, specifically in patients with HER2+, ER/PR-, node-positive tumors, and thus was used in high-risk HER2+ BC management \citep{von2017adjuvant}. Besides mAbs, a small molecule tyrosine kinase inhibitor (TKi), Neratinib, also showed prognostic benefits for ER/PR+ HER2+ patients in a randomized trial \citep{martin2017neratinib}. Taken together, HER2-targeting mAbs or small molecules in combination with chemotherapy benefit both primary and metastatic HER+ breast cancer. 

\subsubsection{Chemotherapy and other systemic therapeutics}

Chemotherapeutic reagents preferentially kill cells which are highly proliferative by multiple proliferation-related manners. Despite low specificity and high toxicity, chemotherapy nevertheless prevents BC recurrence of multiple stages (I - III), and is the only therapy showing effectiveness against TNBCs. BC specific chemo regimen was firstly developed from last 1970s - till now, three major regimes are available - 1) docetaxel / cyclophosphamide (DC), 2) adriamycin / cyclophosphamide (AC) 3) AC + paclitaxel (AC-T). Taxol triggers mitotic arrest by reducing microtubule dynamicity; cyclophosphamide as an alkylating agent generates DNA crosslinks and eventually induces apoptosis; and adriamycin interferes with DNA replication. These three combinations are usually used in metastatic breast cancer, but also apply for primary diseases in cases of highly proliferative ER/PR+/HER-, HER2+, and TNBC tumors. Other reagents are also available as first or later lines in metastatic TNBCs - e.g., platinum, capecitabine, gemcitabine and vinorelbine \citep{waks2019breast}.

Apart from endo / anti-HER2 / chemotherapies, targeted and immune therapies may also utilized, especially in metastatic BC (MBC). MBC usually shows high invasiveness and rapid resistance development, thus stronger combinatorial regimen and sequential administrations (first and later lines) may apply. These include: CDK4/6 inhibitors (abemaciclib, palbociclib, ribociclib) in ER/PR+, HER2- MBC; PARP inhibitors (Olaparib, Talazoparib) for BRCA1/2 mutated diseases; and mTOR inhibitors in ER/PR+ MBCs \citep{waks2019breast}. Additionally, immune checkpoint blockade (ICB), such as PD-1/PD-L1 and CTLA-4 blockade, has been extensively investigated, especially for HER+ BCs and TNBCs which show high immunogenicity \citep{adams2019current}. Combination of Atezolizumab, the PD-L1 antibody, with nabpaclitaxel was approved by FDA in 2019 for metastatic TNBC which show high PD-L1 expression \citep{schmid2018atezolizumab}. In future, more efficacy biomarkers as well as novel strategies are in need for a further precise treatment scheme development, especially in metastatic breast cancers.

\subsection{Breast Cancer Treatment}

\subsubsection{Clinical reccurrence and treatment}

Despite efficient treatment for primary disease, breast cancer may recur - occurring either locally, regionally to lymph nodes or distantly to other organs. Compared to primary tumor, recurrent disease often shows much higher heterogeneity and resistance. In US, 6-10\% of diagnosis is metastatic (http://mbcn.org/incidence-and-incidence-rates/); and even for the most benign primary BC (T1N0, ER/PR+), total recurrence risk can reach around 13\% in 15-20 years \citep{pan201720}. Metastatic (stage IV) BC shows much poorer prognosis than primary - 4-5 years overall survival for ER/PR+ or HER2+ and 1 year for TNBC \citep{bardia2017efficacy,ellis2015fulvestrant,swain2015pertuzumab}.

ER/PR+ BC generally shows late relapse - the 5 year the risk of recurrence (RoR) ranges from ~5-20\% but with steady longitudinal increasement up to 20 years (~20-50\%) . For early stage ER+ BC, both RoR and disease specific death is most predictive by TN stage \citep{pan201720}. Comparatively, HER2+ BC or TNBC recur early, with high RoR within the first five years while decrease afterwards. HER2+ BC has a five-year cumulative relapse hazard of 10.96\% and 17.48\% (chemo or trastuzumab-chemo combined), which steady at 10 years with 5-10 cumulative recurrence/death hazard at 5.75\% and 8.6\%, for ER/PR+/- respectively \citep{chumsri2019incidence}. TNBC has higher distant recurrence rate and death than other subtypes, with annual distant recurrence hazard peaking from 1-4 years (10\%-15\%) but rapidly decrease afterwards - reaching approximately null-recurrence after 8 years \citep{dent2007triple}. 

Distant metastasis is the most common BC recurrence (49\%) \citep{bruce1970patterns}- most preferentially goes to bone, liver, lung and brain, as the ‘seed and soil’ suggests. BC metastasis shows subtype-specific organ tropisms - e.g., ILCs are significantly more likely to metastasize to peritoneum, ovary and gastrointestinal tracts than IDCs \citep{mathew2017distinct}; ER-/PR-,HER2+ more metastasize to brain than ER+/PR+,HER2- and TNBCs mainly to lung (Chen et al., 2018b). Of note, BC subtypes may switch upon metastasis - e.g., 10-30\% tumors exhibit positive to negative conversion of hormone receptors (ER,PR,HER2), and thus re-assessment of metastasis should be considered \citep{schrijver2018receptor}. In general, metastasis is a multi-step process, in which tumor cells dynamically evolve and travel, and eventually establish its niche at distant organ to expand. Different seeding and evolutionary models have been developed for this process, which may help better understanding and treatment. 


\subsubsection{Genetic alterations in BC metastasis}

Genetic/transcriptomic profiles of metastases have been extensively investigated, to illustrate general landscape, identify drivers, and explore druggabilities. The MET500 in 2017 profiled 500 patients with metastatic solid tumors with WES and RNA-seq, which revealed frequently mutated genes (TP53, CDKN2A, PTEN, PIK3CA, RB1), heterogeneous transcriptomic phenotypes (proliferative and EMT-like) and immune states - each suggestive to potential targeting or immune checkpoint blockade \citep{robinson2017integrative}. Razavi et al. revealed features of endocrine resistant BCs by profiling genomes of 1,918 BCs, 692 of which received hormone therapy, and highlighted increased alterations in MAPK pathways genes, ER transcriptional regulators, ERBB2 and NF1 LOF (loss of function) events \citep{razavi2018genomic}.  Bertucci et al. profiled 617 metastatic BCs from six clinical trials with WES, which reported more frequently mutated drivers (TP53, PIK3CA, GATA3, ESR1, KMT2C, CDH1, PTEN, NF1), amplicons (signature 2,3,10,13,17) and increased tumor burden in metastasis \citep{bertucci2019genomic}. Another large scale study of pan-cancer metastasis genome profiling was performed on 2,520 tumor-normal tissue pairs, which illustrated great translation potential - half patients harbor a predicted event targetable by approved anti-cancer drugs; and over half lung/skin cancer bear a high tumor mutation burden suitable for ICB treatment \citep{priestley2019pan}. 

Besides metastasis profiling alone, paired metastatic and primary tumors from the same patient also serve as invaluable resources in delineating cancer evolution. Brastianos et al. observed independent evolution of BC primary versus brain metastasis, and unique actionable alterations (CDKs, HER2/EGFR, PI3K/AKT/mTOR) in brain MBCs \citep{brastianos2015genomic}. Ng et al. revealed significant heterogeneity (a median of 40\% somatic mutational differences) between primary and MBCs even at treatment-naïve conditions, as reflected in mutational signature shifts \citep{ng2017genetic}. Siegel et al. adopted a multi-omics approach to integratively deduce tumor-specific drivers and phylogeny, which suggested early driver establishment in primary tumor, a multiclonal seeding pattern, and enrichment of estrogen/androgen receptors in metastasis-specific drivers \citep{siegel2018integrated}. Ullah et al. reported a minimal involvement of axillary lymph nodes to distant seeding of MBC and thus highlighted prevalence of hematogenous spreading \citep{ullah2018evolutionary}. A latest study from Hu et al. used 457 paired samples from 136 patients of three cancer types (colorectal, lung, breast), revealing that 1) treated metastases bear more potential driver mutations than untreated ones¬¬, 2) monoclonal seedings occur mostly, especially untreated distant metastases than treated ones 3) metastasis may well occur 2-4 years before primary tumor diagnosis, though in case of brca, disseminations show higher latency than lung or colorectal cancers. These findings illustrate the striking chronology of early seeding and emphasize impact of treatment-mediated clonal selection \citep{hu2020multi}. Additionally, our group also made valuable findings utilizing over 100 paired primary and metastatic BC samples, which revealed RET overexpression (Varešlija et al., 2019a) and ERBB2 amplification in BC brain metastasis (Priedigkeit et al., 2017a); unique signatures in BC bone metastasis (Priedigkeit et al., 2017b); FGFR4 overexpression \citep{levine2019fgfr4}, increased hyperactive ESR1 fusions \citep{hartmaier2018recurrent}, and reduced immune recruitment in resistant/metastatic BC \citep{zhu2019metastatic}.


\subsubsection{Strategies tackling metastasis}

Given these many features observed, we may well wonder how to tackle metastasis accordingly. Two answers might apply - timely detection and personalized therapies. 

For the first answer, imaging is the least invasive manner for cancer tracking, but with low sensitivity. In comparison, CTC and cell-free DNA (cfDNA) or other tumor-related materials from regular liquid biopsy could be more informative - lower CTC counts or ctDNA (circulating cell-free tumor DNA) levels indicate worse prognosis \citep{bidard2014clinical}, and either targeted detection (e.g., ddPCR) or whole genome/exome sequencing could be performed on cfDNA to monitor specific pathogenic mutations or structural variants \citep{alimirzaie2019liquid}. E.g., Regarding prognosis, ERBB2 cfDNA fragments were notably relate to increased recurrence risk \citep{maltoni2017cell}, PIK3CA mutation in plasma ctDNA could suggest shorter-recurrence free survival \citep{oshiro2015pik3ca}. Regarding therapy response, ESR1 mutation enrichment identified from cfDNA could well indicate endocrine resistance cohort \citep{beije2018estrogen}; PD-L1 expression in CTCs could monitor ICB efficiency \citep{mazel2015frequent}; and that early ctDNA dynamics could predict CDK4/6 inhibitor responsiveness \citep{o2018early}. Similar practices were performed in our group too, from which we illustrated longitudinal alterations of ESR1 fusion and mutations from plasma cfDNA through endocrine treatment.

Concerning the second answer, a simple solution would be drug development and testing against actionable targets - as has been practiced in cancer for years with impressive success - e.g., imatinib against BCR-ABL fusion in leukemia \citep{an2010bcr}; EGFR inhibitors against mutations in non-small cell lung cancer \citep{gerber2008egfr}; trastuzumab against HER2 \citep{gajria2011her2} and PARP inhibitors against BRCA1/2 mutation in BC \citep{dziadkowiec2016parp}; as well as inhibitors for TKs, CDKs, PI3Ks against tumors in general. However, such approaches are somewhat ‘passive’ -only winning short-term victory while newly-developed resistance may soon occur. Some alternative strategies might solve this dilemma more proactively - immune activation and drug design targeting tumor evolution per se. The first strategy includes ICB, T-cell engineering, and antigen vaccines. It boosts our intrinsic defense system, and thus utilizes such adaptivity to detect and eliminate newly evolved antigens. Immune therapy currently available is anti-PD-L1 mAb (Atezolizumab) in combination with chemotherapy (nab-paclitaxel) in metastatic TNBC, yet other possibilities exist, given high immunogenicity of MBCs. The second strategy generates ‘evolutionary trap’ - either via ‘collateral sensitivity’ (the second drug killing cells resistant to the first), or by ‘competition’ (ceased administration of the first drug restore survival advantage of sensitive cells). Example for the former includes a G>C substitution in ABL1 which causes imatinib resistance, while also sensitizing cells to other non-classical BCR-ABL1 inhibitors \citep{zhao2016exploiting}; and a collective CDKN2A deletion which induces trametinib (MEK inhibitor) resistance but CDKi vulnerabilities in NSCLC \citep{acar2020exploiting}. Practices of the latter is nicely shown by intermittent dosing in mCRPCs (metastatic castrate-resistant prostate cancer), which showed over 10 months extension of progression-free period than standard dosing. Beyond this, one other bold approach aims to target neogenesis of heterogeneity per se, which is the source of evolution - a putative candidate being APOBEC, the inhibitor of which has been under active investigation \citep{barzak2019selective}.

In summary, BC recurrence, as the leading cause of mortality, commonly occurs in the 20-year window post diagnosis, and is often treated with more aggressive therapeutics including more targeting and immune therapy. Metastatic BC has greatly increased heterogeneity than primary, with different clonal evolutionary and seeding patterns observed, and metastasis/resistance driver genes proposed from multiple large-scale sequencing studies. Tackling MBCs calls for early detection of cfDNA or CTCs from liquid biopsy, as well as personalized therapeutics involving both anti-driver drug development and more proactive strategies like immune therapy or evolutionary-based treatments.

\section{Single cell sequencing in cancer research}

\subsection{The decade of single cell sequencing}

The nature of life never fails to intrigue human beings - since discovery of single cells in 1660s, we have been closer to the truth - from observation of morphology to quantification of molecules which control heredity and phenotypes. Notably, high-throughput DNA and RNA quantification - microarrays in 1990s and next-generation of sequencing (NGS) in 2000s - have potentiated life science towards a more scalable and data-driven track, away from previous reliance on accidental discoveries. Cancer research also witnessed several grand NGS guided projects - TCGA (the Cancer Genome Atlas), ICGC (the International Cancer Genome Consortium), CCLE (the Cancer Cell Line Encyclopedia) \citep{ghandi2019next}, DepMap (the Cancer Dependency Map) \citep{tsherniak2017defining} and LINCS (the Library of Integrated Network-Based Cellular Signatures). These large-scale datasets greatly empowered cancer studies and clinical translations.

These achievements are transforming, but not enough. Heterogeneity of life and diseases arises from single molecule or cells while most NGS atlases only reach resolution of bulk tumors. Single cell (SC) sequencing (SCS) thus may best answer this question. Highlighted as `2013 Nature Method of the year'. Untargeted SCS first appeared in 2009, in which the whole transcriptome of a single cell was sequenced \citep{tang2009mrna}, soon followed by scDNA-seq in 2011 \citep{navin2011tumour}, then rapid development of new techniques, commercialization and a boost of applications. The current SC technologies readily capture multiple features, including chromatin conformation, epigenomics, proteomics, and spatial location. SC multimodal omics has again been named as `2019 Nature Method of the year', pointing towards a future of integrative multi-omics. Accordingly, next generation of atlas at single cell resolution have been launched, the human cell atlas \citep{regev2017human}, the Pre-Cancer Atlas Pilot Project, and the human tumor atlas \citep{rozenblatt2020human}, aiming for a comprehensive reference of normal and diseased tissues ranging from the molecular features to clinical manifestations. All these discoveries will gradually transform human’s understanding of ourselves in a deeper and wider manner, and in the form of big data. 

\subsection{SCS techniques and applications}

SCS has mainly two categories - techniques which detect one single `omic', or those measuring multiple-omics simultaneously. The first class largely overlaps with bulk sequencing, e.g., chromosome conformation (scHi-C-seq), DNA sequence (SNS, SCI-seq), DNA methylation (scBS-seq, sci-MET, snmC-seq), chromatin accessibility (scATAC-seq, sciATAC-seq, scTHS-seq), histone modifications (scChIP-seq), or lineage tracing (ScarTrace, LINNAEUS). The second class mainly includes DNA+RNA (G\&T-seq), RNA+surface protein (CITE-seq, REAP-seq), and RNA+spatial information (STARmap, MERFISH), etc (see table below for references). Specifically, SCS can be coupled with genome-editing (e.g., CRISPR-Cas9) as to investigate effects like gene KO. A leap in technique development lies in robustness and scalability - from manual operations of low coverage and throughput to automatic workflow with high throughput. Two strategies are frequently adopted: 1) combinatorial indexing, in which cells are barcoded for multiple rounds to be uniquely identified; and 2) droplet-based microfluidics, in which single cells and barcode-beads were initiated as individual droplets, and one-to-one merged afterwards. A summary of commonly used SCS is shown as below.

\begin{longtable}{{p{2cm} p{2cm} p{2cm} p{3cm} p{1cm} p{2cm} }}
  \caption{SCS Techniques}
  \label{tab:longtable} \\
  \toprule
  Method & Data Type & Cell Throughput & Feature & Time & Reference \\
  \midrule
\endfirsthead
  \caption*{Continued Table~\thetable\quad SCS Techniques} \\
  \toprule
  Method & Data Type & Cell Throughput & Feature & Time & Reference \\
  \midrule
\endhead
  \bottomrule
\endfoot
  scTHS-seq & Chromatin accessibility & 10-20k & droplet-based & 2018 & \citep{lake2018integrative}  \\
  scHi-C-seq & Chromosome conformation & 1-10 & manual isolation of single cells & 2013 & \citep{nagano2013single}  \\
  Sci-Hi-C & Chromosome conformation & 100-2000  & combinatorial-indexing, high-throughput  & 2020 & \citep{ramani2020sci} \\
  Perturb-seq & CRISPR + RNA & 1-10k & droplet-based scRNA-seq combined with pooled CRISPR screening & 2016 & \citep{dixit2016perturb} \\
  SNS & DNA & 10-100 & WGA on flow-sorted nuclei, detect CNV & 2011 & \citep{navin2011tumour} \\
  SCI-seq & DNA & 10-20k & combinatorial-indexing, high-throughput with low cost, detect CNV & 2017 & \citep{vitak2017sequencing} \\ 
  LIANTI & DNA & around 10 & linear amplification of DNA fragments from Tn5 transposition, detect SNV & 2017 & \citep{chen2017single} \\
  SiC-seq & DNA & more than 50k & droplet based, high-throughput & 2017 & \citep{lan2017single} \\
  scRRBS & DNA methylation & 1-10 & detect methylation up to 1.5M CpG sites in mouse embryonic stem cell  & 2013 & \citep{guo2013single} \\
\end{longtable}