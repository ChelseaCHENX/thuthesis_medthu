% !TeX root = ../thuthesis-example.tex

\begin{denotation}[3cm]
  \item[BC]	breast cancer 
  \item[LN]	lymph node
  \item[SUM44]	SUM44-PE cell line
  \item[MM134]	MDA-MB-134 cell line
  \item[GEMM]	genetically engineered mouse model
  \item[FFPE]	formalin-fixed and paraffin-embedded
  \item[TME]	tumor microenvironment
  \item[PAM50]	Prosigna Molecular 50
  \item[ITH]	intra-tumor heterogeneity
  \item[NGS]	next-generation sequencing
  \item[WGS]	whole genome sequencing
  \item[WES]	whole exome sequencing
  \item[Indel]	short insertion and deletion in genome
  \item[CNA]	copy number alteration
  \item[LOH]	loss of heterozygosity
  \item[3C]	chromosome conformation capture
  \item[ChIP-seq]	chromatin immunoprecipitation sequencing
  \item[WGBS]	whole-genome bisulfite sequencing
  \item[ATAC-seq]	Assay for Transposase-Accessible Chromatin using sequencing
  \item[RNA-seq]	RNA-sequencing
  \item[scRNA-seq]	single cell RNA sequencing
  \item[CyTOF]	mass cytometry
  \item[FACS]	fluorescence-activated cell sorting 
  \item[PCA]	principle component analysis
  \item[t-SNE]	t-Distributed Stochastic Neighbor Embedding
  \item[UMAP]	Uniform Manifold Approximation and Projection
  \item[CRISPR]	clustered regularly interspersed short palindromic repeat
  \item[CAS9]	CRISPR-associated protein 9
  \item[IHC]	immunohistochemistry
  \item[RT-PCR]	reverse transcription polymerase chain reaction
  \item[qPCR]	quantitative polymerase chain reaction
  \item[WB]	western blot
  \item[Dual-Luc]	dual-luciferase reporter assay
  \item[CTD2]	Cancer Target Discovery and Development
\end{denotation}



% 也可以使用 nomencl 宏包,需要在导言区
% \usepackage{nomencl}
% \makenomenclature

% 在这里输出符号说明
% \printnomenclature[3cm]

% 在正文中的任意为都可以标题
% \nomenclature{PI}{聚酰亚胺}
% \nomenclature{MPI}{聚酰亚胺模型化合物,N-苯基邻苯酰亚胺}
% \nomenclature{PBI}{聚苯并咪唑}
% \nomenclature{MPBI}{聚苯并咪唑模型化合物,N-苯基苯并咪唑}
% \nomenclature{PY}{聚吡咙}
% \nomenclature{PMDA-BDA}{均苯四酸二酐与联苯四胺合成的聚吡咙薄膜}
% \nomenclature{MPY}{聚吡咙模型化合物}
% \nomenclature{As-PPT}{聚苯基不对称三嗪}
% \nomenclature{MAsPPT}{聚苯基不对称三嗪单模型化合物,3,5,6-三苯基-1,2,4-三嗪}
% \nomenclature{DMAsPPT}{聚苯基不对称三嗪双模型化合物(水解实验模型化合物)}
% \nomenclature{S-PPT}{聚苯基对称三嗪}
% \nomenclature{MSPPT}{聚苯基对称三嗪模型化合物,2,4,6-三苯基-1,3,5-三嗪}
% \nomenclature{PPQ}{聚苯基喹噁啉}
% \nomenclature{MPPQ}{聚苯基喹噁啉模型化合物,3,4-二苯基苯并二嗪}
% \nomenclature{HMPI}{聚酰亚胺模型化合物的质子化产物}
% \nomenclature{HMPY}{聚吡咙模型化合物的质子化产物}
% \nomenclature{HMPBI}{聚苯并咪唑模型化合物的质子化产物}
% \nomenclature{HMAsPPT}{聚苯基不对称三嗪模型化合物的质子化产物}
% \nomenclature{HMSPPT}{聚苯基对称三嗪模型化合物的质子化产物}
% \nomenclature{HMPPQ}{聚苯基喹噁啉模型化合物的质子化产物}
% \nomenclature{PDT}{热分解温度}
% \nomenclature{HPLC}{高效液相色谱(High Performance Liquid Chromatography)}
% \nomenclature{HPCE}{高效毛细管电泳色谱(High Performance Capillary lectrophoresis)}
% \nomenclature{LC-MS}{液相色谱-质谱联用(Liquid chromatography-Mass Spectrum)}
% \nomenclature{TIC}{总离子浓度(Total Ion Content)}
% \nomenclature{\textit{ab initio}}{基于第一原理的量子化学计算方法,常称从头算法}
% \nomenclature{DFT}{密度泛函理论(Density Functional Theory)}
% \nomenclature{$E_a$}{化学反应的活化能(Activation Energy)}
% \nomenclature{ZPE}{零点振动能(Zero Vibration Energy)}
% \nomenclature{PES}{势能面(Potential Energy Surface)}
% \nomenclature{TS}{过渡态(Transition State)}
% \nomenclature{TST}{过渡态理论(Transition State Theory)}
% \nomenclature{$\increment G^\neq$}{活化自由能(Activation Free Energy)}
% \nomenclature{$\kappa$}{传输系数(Transmission Coefficient)}
% \nomenclature{IRC}{内禀反应坐标(Intrinsic Reaction Coordinates)}
% \nomenclature{$\nu_i$}{虚频(Imaginary Frequency)}
% \nomenclature{ONIOM}{分层算法(Our own N-layered Integrated molecular Orbital and molecular Mechanics)}
% \nomenclature{SCF}{自洽场(Self-Consistent Field)}
% \nomenclature{SCRF}{自洽反应场(Self-Consistent Reaction Field)}
